Questo progetto si pone l'obiettivo di sviluppare un editor di scene virtuali accessibile via Web utilizzando delle tecnologie emergenti quali X3DOM, frutto di un processo evolutivo e di standardizzazione iniziato circa venti anni fa, al fine di  verificarne le potenzialit\'{a} e comprendere il grado di maturit\'{a} raggiunto. 

Il fine ultimo di questo studio \`{e} quello di verificare la possibilit\'{a} di applicare i risultati raggiunti nel presente lavoro di Tesi, nell'ambito della riabilitazione  di pazienti affetti da lesioni cerebrali gravi. Nel gruppo di ricerca nel quale \`{e} stata svolta l'attivit\'{a} di Tesi, infatti, \`{e} attiva una linea di ricerca\cite{VirtualReality} orientata all'applicazione di tecniche di realt\'{a} virtuale combinate con ausili e dispositivi riabilitativi al fine di migliorare ed accelerare il recupero funzionale dei pazienti con lesioni neurologiche gravi. Nello specifico l'ambiente di editing dovrebbe consentire al terapista di assemblare la scena virtuale relativa ad un supermercato, ponendo negli scaffali oggetti di vario tipo, possibilmente ai quali il paziente \`{e} abituato e che conosce molto bene. Una volta assemblata la scena e selezionato il tipo di attivit\'{a} da assegnare al paziente, la scena virtuale viene salvata e resa disponibile al paziente nel corso della sessione riabilitativa.

Nel predisporre la scena virtuale il terapista dispone di una libreria di oggetti realizzati con il linguaggio X3D,  che pu\'{o} arricchire corredandolo di nuovi oggetti, possibilmente molto familiari al paziente. Utilizzando l'ambiente di editing, il terapista sar\'{a} facilitato nel disporre gli oggetti nelle posizioni desiderate, rendendo possibile la creazione di scene virtuali sempre pi\'{u} vicine all'esperienza ed alla storia vissuta dal paziente.

Il paziente,  nello svolgere la sessione riabilitativa, si collegher\'{a} attraverso una stazione dotata degli ausili opportuni, come ad esempio Nu!Reha Desk\cite{nurDesk}, al sito internet che ospita il piano di esercizi riabilitativi da svolgere e procede con lo svolgimento dell'esercizio. I dati inerenti l'esercizio svolto vengono trasmessi al Centro di Riabilitazione, dove i medici potranno monitorare le attivit\'{a} svolte dal paziente ed i progressi effettuati.

L'avvento di X3DOM, che incarna i principi pi\'{u} avanzati dello standard HTML 5 e permette di fruire efficacemente di applicazioni 3D tramite Web, eliminando la necessit\'{a} di software addizionali specifici e permettendo al programmatore di definire scenari con una complessa gamma di possibilit\'{a}, rappresenta una tecnologia innovativa ed estremamente promettente.

%Si pensi ad un supermercato. Si immagini questo ambiente riprodotto in uno scenario virtuale. Due sono gli attori principali in gioco: il cliente, che si muove all'interno dell'ambiente, osserva e sceglie tra gli oggetti esposti cosa acquistare, e il gestore, che si occupa di predisporre ed organizzare gli scaffali con i prodotti a disposizione. Il gestore dunque configura la scena per il cliente. In uno spazio virtuale il gestore necessita di uno strumento, un editor che gli consenta di inserire, posizionare ed eliminare gli oggetti (in questo caso i prodotti del supermercato) e di salvare le proprie modifiche rendendole disponibili all'utente.

Il caso d'uso descritto delinea un'applicazione di tipo gestionale, in cui diversi utenti accedono con vari livelli di autorizzazione ai servizi messi a disposizione. \`{E} facile pensare da subito ad un software per il desktop come alla soluzione pi\'{u} naturale e semplice da implementare, vista anche la necessit\'{a} di rappresentare una scena 3D con la quale l'utente possa interagire. Negli ultimi anni si \'{e} per\'{o} assistito ad un fiorire di applicazioni Web  sempre pi\'{u} complesse, dalle ``office suite'' fino ai gestionali per aziende e programmi per il photo ed il video editing.

Seguendo perci\'{o} questo trend di migrazione delle applicazioni dal desktop al web, si fa strada l'idea di avere un editor 3D come quello descritto interamente in un browser. Possibilit\'{a} questa interessante per tutti i vantaggi che un'applicazione web porta con s\'{e}, quali la compatibilit\'{a} cross-platform, nessuna installazione e update lato utente grazie all'utilizzo del browser come thin-client, facilit\'{a} di integrazione con servizi web e possibilit\'{a} di funzionalit\'{a} avanzate (ad esempio l'editing collaborativo). Sorgono per\'{o} alcune domande. Quali tecnologie ci sono a disposizione per il 3D nel web oggi? Sono abbastanza mature per realizzare un'applicazione affidabile e performante? Quali svantaggi presentano? 

Il compito di trovare una risposta a queste domande \`{e} affidato alla prima parte di questo lavoro. Dapprima verranno introdotte le possibili soluzioni presenti nell'era ante Web2.0 quali i linguaggi VRML prima e X3D poi, integrati nei browser tramite plugins di terze parti. Si analizzeranno in seguito i motivi per i quali queste non hanno avuto il successo sperato. Da questo piccolo fallimento si vedr\'{a} poi come nasce l'idea di un web 3d senza plugin, ``nativo'', e quali tecnologie lo rendono possibile. Si parler\'{a} dunque della specifica HTML5, ancora in via di definizione che grazie alle novit\'{a} introdotte pone le basi per il 3D nel web, di WebGL, un nuovo standard promosso dal Khronos Group che definisce un interfaccia di programmazione di basso livello per la creazione di contenuti 3D all'interno di un browser web, e di X3DOM, un motore Javascript che consente l'introduzione di elementi X3D come parte del DOM di una pagina HTML5 e che fa uso dell'API WebGL per visualizzarli.

L'unico modo di testare efficacemente questo sempre pi\'{u} fiorente intreccio di tecnologie \`{e} effettuare una prova sul campo. Il software Medea, prodotto in questo lavoro, nasce esattamente a questo scopo. Nella seconda parte si analizzeranno le specifiche funzionali e le caratteristiche tecniche del software. Si vedr\'{a} come sono state utilizzate ed interconnesse le tecnologie di cui sopra per raggiungere gli obiettivi prefissati. Si cercher\'{a} di valutare l'efficienza e l'usabilit\'{a} della soluzione proposta, al fine di individuare e comprendere quali siano i pregi e quali i difetti e i limiti di questa implementazione di 3D nel web. Obiettivo finale \`{e} comprendere se queste tecnologie siano pronte e mature per supportare un'applicazione di produzione completa, performante, cross-platform e user-friendly.
